\documentclass[a4paper,14pt]{article}
\usepackage[paperwidth=210mm,paperheight=297mm,centering,hmargin=2cm,vmargin=2.5cm]{geometry}
\usepackage{epsfig}
\usepackage{csquotes}
\usepackage{gensymb}
\usepackage{amsmath}
\usepackage{amssymb}
\usepackage{appendix}
\usepackage{color}
\definecolor{darkblue}{rgb}{0.0,0.0,0.4}
\usepackage{csquotes}
\usepackage{lineno}
\usepackage{authblk}
\usepackage[numbers,comma,super,sort&compress]{natbib}
\usepackage[colorlinks=true,linkcolor=blue]{hyperref}%
\makeatletter 
\renewcommand\@biblabel[1]{#1.} 
\makeatother
\hypersetup{
    colorlinks=true,
    linkcolor=darkblue,
    filecolor=magenta,      
    urlcolor=darkblue,
}

\begin{document}
\noindent\textbf{\Large Swaminath Bharadwaj G.}\\
\rule{\textwidth}{1pt}\\\ \\
{\normalsize
\noindent Postdoctoral Researcher\\
Computational Physical Chemistry\\
Eduard-Zintl-Institut f{\"u}r Anorganische und Physikalische Chemie\\
Technical University of Darmstadt \\
Darmstadt-64287, Germany\\
\textbf{Email:} swaminath.bharadwaj@tu-darmstadt.de, swaminath6989@gmail.com\\
\textbf{Phone:} +49 174 5768580\\
\textbf{Homepage:} \href{https://swaminath6989.gitlab.io/swaminath6989/}{https://swaminath6989.gitlab.io/swaminath6989/}\\
}
\section*{\large Education}
\begin{tabular}{l p{14cm}}
2011-2018&Integrated M.Tech-Ph.D, Department of Chemical Engineering\\
& Indian Institute of Technology Madras, Chennai, India\\
& \textbf{Advisors}: Prof. Abhijit P. Deshpande, Prof. P. B. Sunil Kumar\\
& \textbf{Thesis}: Coarse Grained Simulation and Mean-Field Modeling of LCST in Thermoresponsive Polymer Solutions\\
&\\
2007-2011&B.Tech, Department of Chemical Engineering\\
& Alagappa College of Technology, Anna University, Chennai, India\\
& G.P.A: 8.91 (scale of 10)
\end{tabular}
\section*{\large Work Experience}
\begin{tabular}{l p{14cm}}
2018-present& Post-doctoral Researcher, Computation Physical Chemistry\\
& Eduard-Zintl-Institut f{\"u}r Anorganische und Physikalische Chemie\\
& Technical University of Darmstadt, Germany\\
& \textbf{Advisor}: Prof. Nico F. A. van der Vegt\\
\end{tabular}
\section*{\large Research Interests}
My research focuses on the modeling of \textbf{stimuli-responsive materials}
and \textbf{interfacial phenomena in soft matter systems} using a combination
of molecular simulations, free energy calculations, statistical mechanical and
mean-field theoretical approaches. A complimentary focus of my research is to
develop methods for the thermodynamic characterization of solute solvation
shell, particularly for flexible solutes such as polymer and biopolymers, using
a combination of molecular simulations, advanced sampling methods and
theoretical frameworks such as small system thermodynamics.
 \section*{\large Publications}
\begin{enumerate}
\item
\enquote{Nonadditive ion effects on the coil-globule equilibrium of PNIPAM: A computer simulation study} Y. Zhao, \underline{S. Bharadwaj}$^{\ast}$, N. F. A. van der Vegt. Accepted in Physical Chemistry Chemical Physics, April 8, 2022 .\\
$^{\ast}$Joint first author
\item
\enquote{Solvation Shell Thermodynamics of Extended Hydrophobic Solutes in
Mixed Solvents} M. Tripathy, \underline{S. Bharadwaj}$^{\ast}$, Nico F. A. van
der Vegt. Accepted in The Journal of Chemical Physics, 1st April, 2022.\\
DOI:\href{https://doi.org/10.1063/5.0090646}{10.1063/5.0090646}\\
$^{\ast}$Joint first author
\item
\textbf{Review}:\enquote{Cononsolvency of thermoresponsive polymers: where we are now and where we are going} \underline{S. Bharadwaj}, B. J. Niebuur, K. Nothdurft, W. Richtering, N. F. A. van der Vegt, C. M. Papadakis. Accepted in Soft Matter, 2022.\\
DOI: \href{https://doi.org/10.1039/D2SM00146B}{10.1039/D2SM00146B}
\item
\enquote{Small-to-large length scale transition of TMAO interaction with hydrophobic solutes}
A. Folberth, \underline{S. Bharadwaj}, N. F. A. van der Vegt,
Physical Chemistry Chemical Physics 24 (4), 2080-2087, 2022. \\DOI: \href{https://doi.org/10.1039/D1CP05167A}{10.1039/D1CP05167A}
\\ 
\textbf{2022 PCCP HOT Articles}
\item
\enquote{Direct Calculation of Entropic Components in Cohesive Interaction Free Energies}
\underline{S. Bharadwaj}, S. Jabes B., N. F. A.  van der Vegt,
The Journal of Physical Chemistry B 125 (39), 11026-11035, 2021.\\
DOI: \href{https://doi.org/10.1021/acs.jpcb.1c05748}{10.1021/acs.jpcb.1c05748}\\
\textbf{Part of special issue \enquote{Dor Ben-Amotz Festschrift}}
\item
\enquote{An interplay of excluded-volume and polymer–(co) solvent attractive interactions regulates polymer collapse in mixed solvents}
\underline{S. Bharadwaj}, D. Nayar, C. Dalgicdir, N. F. A van der Vegt,
The Journal of Chemical Physics 154 (13), 134903, 2021.\\
DOI: \href{https://doi.org/10.1063/5.0046746}{10.1063/5.0046746}
\item
\textbf{Communication}: \enquote{A cosolvent surfactant mechanism affects polymer collapse in miscible good solvents}
\underline{S. Bharadwaj}, D. Nayar, C. Dalgicdir, N. F. A. van der Vegt,
Communications Chemistry 3 (1), 1-7, 2020.
DOI: \href{https://doi.org/10.1038/s42004-020-00405-x}{10.1038/s42004-020-00405-x}
\item
\enquote{Characterizing Polymer Hydration Shell Compressibilities with the Small-System Method}
M. Tripathy, \underline{S. Bharadwaj}$^{\ast}$, S. Jabes B., N. F. A. van der Vegt,
Nanomaterials 10 (8), 1460, 2020.\\
DOI: \href{https://doi.org/10.3390/nano10081460}{10.3390/nano10081460}\\
$^{\ast}$Joint First Author\\
\textbf{Part of the special issue \enquote{Nanoscale Thermodynamics}}
\item
\enquote{Does preferential adsorption drive cononsolvency?}
\underline{S. Bharadwaj}, N. F. A van der Vegt.
Macromolecules 52 (11), 4131-4138, 2019.\\
DOI: \href{https://doi.org/10.1021/acs.macromol.9b00575}{10.1021/acs.macromol.9b00575}
\item
\enquote{Kosmotropic effect leads to LCST decrease in thermoresponsive polymer solutions}
\underline{S. Bharadwaj}, P. B. Sunil Kumar, S. Komura, A. P. Deshpande,
The Journal of Chemical Physics 148 (8), 084903, 2018.\\
DOI: \href{https://doi.org/10.1063/1.5012838}{10.1063/1.5012838}
\item
\enquote{Spherically symmetric solvent is sufficient to explain the LCST mechanism in polymer solutions}
\underline{S. Bharadwaj}, P. B. Sunil Kumar, S. Komura, A. P Deshpande, Macromolecular Theory and Simulations 26 (2), 1600073.\\
DOI: \href{https://doi.org/10.1002/mats.201600073}{10.1002/mats.201600073}
\end{enumerate}
\section*{Conferences}
\subsection*{\normalsize Oral Contributions}
\begin{enumerate}
\item
\enquote{Nonadditive ion effects on the solvation of thermoresponsive polymers
in aqueous solutions,}\\ \underline{Swaminath Bharadwaj} , Yani Zhao, Nico F. A.
van der Vegt. \textit{CompFlu}, Indian Institute of Technology Gandhinagar,
India, December 13-15, 2021.
\item
\enquote{A cosolvent surfactant mechanism affects polymer collapse in miscible
good solvents,} \\\underline{Swaminath Bharadwaj}. \textit{Osmolyte and Cosolvent
Effects in Stimuli-Responsive Soft Matter Systems}, Virtual conference
organized jointly by Technical University of Darmstadt and Technical University
of Munich, February 25-26, 2021.
\item
\enquote{On the mechanism of lower critical solution temperature and its
variation with co-solvents in thermoresponsive polymer solutions,}
\underline{Swaminath Bharadwaj}, P. B. Sunil Kumar, Shigeyuki Komura, Abhijit
P. Deshpande. \textit{CompFlu}, Indian Institute of Technology Madras,  Chennai,
India, December 18-20, 2017.
\item
\enquote{Effect of co-solvent on the LCST of thermoresponsive
polymers,} \underline{Swaminath Bharadwaj}, P. B. Sunil Kumar, Shigeyuki Komura,
Abhijit P. Deshpande. \textit{2nd International Conference on Soft Materials
(ICSM)}, Malaviya National Institute of Technology Jaipur, India, December
12-16, 2016. \textbf{Best Oral Presentation Award}
\end{enumerate}
\subsection*{\normalsize Poster Contributions}
\begin{enumerate}
\item
\enquote{Characterizing Polymer Hydration Shell Thermodynamics Using the Small
System Method,}\\  \underline{Swaminath Bharadwaj}, Nico van der Vegt.
\textit{Recent progress in the statistical mechanics of solutions through
Kirkwood-Buff integrals and related approaches}, Universit{\'e} de Bourgogne,
France, September 20-22, 2021
\item
\enquote{Coil-to-globule transitions in mixed solvents: A cosolvent surfactant
mechanism,} \underline{Swaminath Bharadwaj}, Nico van der Vegt. \textit{11th
Liquid Matter Conference}, Prague, July 19-23, 2021
\item
\enquote{A cosolvent surfactant mechanism affects polymer collapse in miscible
good solvents,}\\ \underline{Swaminath Bharadwaj}, Nico van der Vegt.
\textit{CompFlu}, Indian Institute of Technology Bombay, India, December 10-12,
2020.
\item
\enquote{On the mechanism of lower critical solution temperature and its
variation with co-solvents in thermoresponsive polymer
solutions,} \underline{Swaminath Bharadwaj}, P. B. Sunil Kumar, Shigeyuki
Komura, Abhijit P. Deshpande. \textit{10th Liquid Matter Conference}, Ljubljana,
Slovenia, July 17-21, 2017.
\item
\enquote{LCST behavior of thermoresponsive polymers in binary solvent
mixtures,} \underline{Swaminath Bharadwaj}, P. B. Sunil Kumar, Shigeyuki
Komura, Abhijit P. Deshpande. \textit{CompFlu}, Indian Institutes of Science
Education and Research, Pune, India, January 2-4, 2016.
\item
\enquote{Coil-to-globule transition in thermoresponsive polymer solutions,}
\underline{Swaminath Bharadwaj}, P. B. Sunil Kumar, Shigeyuki Komura, Abhijit
P. Deshpande. \textit{Physics of Structural and Dynamical Hierarchy in Soft
Matter}, Institute of Industrial Science, University of Tokyo, Japan, March
16-18, 2015.
\end{enumerate}
\subsection*{Organization}

\begin{enumerate}
\item
Organization of the virtual conference \enquote{Osmolyte and cosolvent effects
in stimuli-responsive soft matter systems} in February, 2021 with Prof. Nico
van der Vegt, TU Darmstadt and Prof. Christine M. Papadakis, TU Munich\\
Website:
\href{https://sites.google.com/view/ocesrsms2021/}{https://sites.google.com/view/ocesrsms2021/}
\item
Assisted in the organization of the conference \enquote{Complex Fluids}, IIT
Madras and IIT Palakkad in 2017 
\item
Initiated and organized the yearly \enquote{In-house Symposium} in the
Department of Chemical Engineering, IIT Madras 
\end{enumerate}
\section*{Research Highlights}

\paragraph{\textbf{Cosolvent Effects on the Phase Behavior of Responsive Polymers:}}
The coil–globule transition of aqueous polymers is of profound significance in
understanding the structure and function of responsive soft matter. In
particular, the remarkable effect of amphiphilic cosolvents (e.g., alcohols)
that leads to both swelling and collapse of stimuli-responsive polymers has
been hotly debated in the literature, often with contradictory mechanisms
proposed. A predominant focus in the literature has been placed on the role of
polymer–cosolvent attractive interactions whereas the role of \textit{excluded
volume interactions (repulsive interactions)} has been largely neglected.
Using molecular dynamics simulations and free energy calculations, we
demonstrated that alcohols \textit{preferentially adsorb} on the polymer and
reduce the free energy cost of creating a repulsive polymer–solvent interface
via a \textit{surfactant-like mechanism} which surprisingly \textit{drives
polymer collapse} at low alcohol concentrations. This hitherto neglected role
of interfacial solvation thermodynamics is common to all coil–globule
transitions, and rationalized the experimentally observed effects of higher
alcohols and polymer molecular weight on the coil-to-globule transition of
thermoresponsive polymers. Further, we showed that polymer–(co)solvent
attractive interactions reinforce or compensate this mechanism and it is this
interplay which drives polymer swelling or collapse. The surfactant-like
mechanism proposed in our study is generic and applicable to other polymer
solutions containing amphiphilic cosolvents.

\paragraph{\textbf{Salt Effects on the Coil-to-Globule Transitions of
Thermoresponsive Polymers:}} The anion Hofmeister series classifies anions in
order of their ability to salt out proteins: CO$_3^{2-}$ $>$ SO$_4^{2-}$ $>$
S$_2$O$_3^{2-}$ $>$ H$_2$PO$_4^{-}$ $>$ F$^{-}$ $>$ Cl$^{-}$ $>$ Br$^{-}$
NO$_3^{-}$ $>$ I$^{-}$ $>$ ClO$_4^{-}$ $>$ SCN$^{-}$.  Specifically, anions on
the right hand side of this series are \textit{weakly hydrated} and
\textit{partition to nonpolar surfaces} such as polymer/water or air-water
interfaces  On the other hand, \textit{strongly hydrated} anions (left hand
side of the series) tend to \textit{strongly interact with water} molecules and
remain in the bulk water environment. Recent experiments have indicated that
the combined effect of a weakly hydrated and a strongly hydrated anion on the
lower critical solution temperature (LCST) of
poly(N-isopropylacrylamide)(PNIPAM) is nonadditive. Our study focused on this
nonadditive effect on the coil-to-globule of PNIPAM using large scale atomistic
simulations. The simulations were able to capture the experimentally observed
effect of mixed salts (NaI-Na$_{2}$SO$_{4}$) on the coil-globule equilibrium of
PNIPAM. Interestingly, the predicted change in LCST as a function of salt
concentration from simulations, through a semi-quantitative analysis, is in
quantitative agreement with experimental measurements. We showed that
nonadditive ion effects on the coil-to-globule transitions of PNIPAM arise due
to the interplay between \textit{depletion of the strongly hydrated sulfate
ions} and the preferential accumulation of the iodide ions on the polymer
leading to \textit{favourable {\rm PNIPAM}-{\rm I}$^{-}$ interactions}. This
correlates with the \textit{partitioning of the {\rm Na}$^{+}$ cations} from
the counterion cloud of the weakly hydrated iodide ions to the counterion cloud
of the strongly hydrated sulfate ions. The proposed mechanism is applicable to
other solutions containing a mixture of a weakly hydrated and strongly hydrated
anion.

\paragraph{\textbf{Hydrophobic Length Scale Dependence of Osmolyte Effects:}}
Osmolytes such as Trimethylamine N-oxide (TMAO) regulate the thermodynamic
stability of proteins and polymers, and are shown to play an important role in
biological systems. The effect of TMAO on the solvation of nonpolar solutes is
not well understood. Our study focused on the effect of TMAO on the
hydrophobic hydration and hydrophobic interactions of \enquote{small} and
\enquote{large} hydrophobic solutes using  molecular dynamics (MD) simulations
and free-energy calculations. Interestingly, the simulation data indicated the
occurrence of a \textit{length scale crossover} in the TMAO interaction with
repulsive Weeks–Chandler–Andersen (WCA) solutes: while TMAO is
\textit{depleted} from the hydration shell of a \textit{small WCA solute}
(methane) and increases the free-energy cost of solute-cavity formation, it
\textit{preferentially binds} to a \textit{large WCA solute} ($\alpha$-helical
polyalanine), reducing the free-energy cost of solute-cavity formation via a
\textit{surfactant-like mechanism}. Significantly, the study shows that this
surfactant-like behaviour of TMAO reinforces the solvent-mediated attraction
between large WCA solutes by means of an entropic force linked to the
interfacial accumulation of TMAO.   It therefore \textit{favours} solute–solute
contact states that \textit{minimise the surface area} exposed to the solvent
and have a small overall number of TMAO molecules adsorbed.

\paragraph{\textbf{Solvation Shell Thermodynamics:}}
The ability of various cosolutes and cosolvents, in aqueous solutions,  to
enhance or quench interfacial solvent density fluctuations have crucial
implications on the conformational equilibrium of macromolecules such as
polymers and proteins. Therefore characterizing the thermodynamics of the
solvation shell and their dependence on stimuli such as cosolvent/cosolute
concentration is very important. The small system method (SSM) exploits the
unique nature of finite sized open systems in which, \textit{thermodynamic
quantities scale with the inverse system size}.  This scaling allows for the
\textit{accurate estimation of properties} in the thermodynamic limit in finite
size simulations.  Our study extends the SSM to characterize the thermodynamics
of the hydration shell of a model extended hydrophobic solute in water and
water-urea/methanol mixtures. In parallel, our other study was aimed at
computing the contribution to the solvation free energy arising from
fluctuating solute-solvent cohesive interactions which is known as
\enquote{Fluctuation entropy}. The fluctuation entropy is usually computed
through \textit{indirect or approximate methods} which are not applicable to
solutes with \textit{flexible conformational degrees of freedom} such as
macromolecules. This study proposed a direct method based on the indirect
umbrella sampling (INDUS) method to \textit{directly compute the fluctuation
entropy} which can be applied to macromolecules.
\section*{Computational Skills}
\begin{itemize}
\item
\textbf{Molecular Simulations}\newline
GROMACS, LAMMPS and PLUMED
\end{itemize}
\begin{itemize}
\item
\textbf{Free energy methods}\newline
Thermodynamic Integration, Free energy perturbation, Umbrella sampling
\end{itemize}
\begin{itemize}
\item
\textbf{Programming Languages }\newline
Extensive use of C, bash and awk for writing scripts and analysis codes. Knowledge
of \LaTeX~for document preparation.
\end{itemize}
\begin{itemize}
\item
\textbf{Related computational software}\newline
VMD, Gnuplot, Matplotlib and Git. Extensively worked with Linux OS.
\end{itemize}


\section*{Teaching Experience}
Over the course of my PhD and postdoc tenure, I have had the opportunity to be
a formal teaching assistant for different courses. My responsibilities included
preparation of weekly exercises, holding problem solving sessions for
students, preparation and correction of written examinations and taking part in
oral examinations. In some of the courses, I had the privilege to deliver one
or two lectures. 
\begin{enumerate}
\item
Teaching assistant with Prof. Nico van der Vegt for the graduate course
\enquote{Physical Chemistry of Soft Matter} from October, 2021 - February, 2022
\item
Teaching assistant with Prof. Nico van der Vegt for the undergraduate course
\enquote{Computer Applications in Chemistry} from April-July, 2021  
\item
Teaching assistant with Prof. Nico van der Vegt for the graduate course
\enquote{Statistical Thermodynamics} from April-July, 2019 
\item
Teaching assistant with Prof. Abhijit P. Deshpande for the graduate course
\enquote{Rheology of Complex Materials} from July-November, 2016 and
July-November, 2017
\item
Teaching assistant with Prof. P. B. Sunil Kumar for the graduate course
\enquote{Methods of Computational Physics} from July-November, 2014
\item
Teaching assistant with Prof. P. B. Sunil Kumar for the undergraduate course \enquote{Statistical Physics and Applications} from July-November, 2013
\end{enumerate}
\section*{References}
\begin{itemize}
\item
\textbf{Prof. Nico F. A. van der Vegt}\newline
Eduard-Zintl-Institut f{\"u}r Anorganische und Physikalische Chemie, Technical University of Darmstadt, Darmstadt, Germany\newline
E-mail: vandervegt@cpc.tu-darmstadt.de
\item
\textbf{Prof. Abhijit P. Deshpande}\newline
Department of Chemical Engineering, Indian Institute of Technology Madras, Chennai, India\newline
E-mail: abhijit@iitm.ac.in
\item
\textbf{Prof. P. B. Sunil Kumar}\newline
Department of Physics, Indian Institute of Technology Palakkad, Ahalia Integrated Campus, Kozhippara, Palakkad, India\newline
E-mail: sunil@iitpkd.ac.in
\item
\textbf{Dr. Shigeyuki Komura}\newline
Wenzhou Institute, University of Chinese Academy of Sciences, Wenzhou, Zhejiang, China\newline
E-mail: komura@wiucas.ac.cn
\item
\textbf{Prof. Walter Richtering}\newline Institut f{\"u}r Physikalische Chemie,
RWTH Aachen University, Germany\newline E-mail: richtering@rwth-aachen.de
\item
\textbf{Prof. Christine M. Papadakis}\newline Department of Physics,
Technical University of Munich, Germany\newline E-mail: papadakis@tum.de
\end{itemize}
\end{document}

